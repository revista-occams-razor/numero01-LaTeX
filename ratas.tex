% Este fichero es parte del N�mero 1 de la Revista Occam's Razor
% Revista Occam's Razor N�mero 1
%
% (c) 2007, 2009, Occam's Razor.
%
% Esta obra est� bajo una licencia Reconocimiento 3.0 Espa�a de
% Creative Commons. Para ver una copia de esta licencia, visite
% http://creativecommons.org/licenses/by/3.0/es/ o envie una carta a
% Creative Commons, 171 Second Street, Suite 300, San Francisco,
% California 94105, USA. 

% Seccion Ratas de Biblioteca
%
% Incluye imagen del art�culo
\rput(1,-2){\resizebox{!}{4cm}{{\epsfbox{zlib_ddj.eps}}}}

% -------------------------------------------------
% Cabecera
\begin{flushright}
\msection{introcolor}{black}{0.25}{RATAS DE BIBLIOTECA}

\mtitle{7cm}{Si no vas sobrao ... -lz}

\msubtitle{12cm}{Como manejar ficheros comprimidos en tus programas}

{\sf por Er aplastao}

{\psset{linecolor=black,linestyle=dotted}\psline(-12,0)}
\end{flushright}

\vspace{2mm}
% -------------------------------------------------

\begin{multicols}{2}

% Definici�n de colores
% FIXME: La definici�n en portada.tex no propaga.
\definecolor{introcolor}{rgb}{0.6,0.7,0.3}
\definecolor{titlecolor}{rgb}{0.4,0.5,0.1}
\definecolor{excolor}{rgb}{0.8,0.8,0.8}

% Introducci�n
\colorbox{introcolor}{
\begin{minipage}{.9\linewidth}

{{\resizebox{!}{1.0cm}{T}}{odos estamos acostumbrados a comprimir
ficheros o incluso directorios enteros cuando empezamos a ir escasos de
espacio en disco. Normalmente, cuando queremos recuperar los datos
comprimidos, primero los descomprimimos y luego los usamos. No ser�a
estupendo eliminarnos este paso?
}}

\end{minipage}
}

\vspace{2mm}

% Cuerpo del art�culo

Muchos habr�is comprobado que en los sistemas GNU/Linux existen varios
programas capaces de trabajar directamente con ficheros comprimidos,
como por ejemplo gv (visor de postscript) o vim (editor de
textos). Lejos de lo que podr�a parecer a simple vista, a�adir esta
funcionalidad a nuestros programas es mucho m�s f�cil de lo que nos
imaginamos gracias a la librer�a {\em libz.so}.

\sectiontext{white}{black}{GRABANDO FICHEROS}

El siguiente fragmento de c�digo muestra como generar un fichero
comprimido utilizando la librer�a {\em zlib}:

{\small
\begin{verbatim}
#include <zlib.h>

int main ()
{
	gzFile  f;
	f = gzopen (``mi_fichero.txt.gz'', ``wt'');
	for (int i = 0; i < 20; i++)
	  gzprintf (f, ``Hello World %d\n'', i);
	gzclose(f);

	return 0;
}
\end{verbatim}
}

En el ejemplo anterior se han omitido todas las comprobaciones de
errores, para poder concentrarnos en el uso de la librer�a. La verdad
es que cualquiera que haya escrito un programa para grabar un fichero
de texto en C lo ver� claro :).

Para compilar este ejemplo, debemos indicar que se utilice la librer�a
zlib, esto lo conseguimos con una l�nea como la siguiente:

\begin{verbatim}
gcc mi_codigo_de_lamuerte.c -o z_ejemplo -lz
\end{verbatim}

Que nos generar� un ejecutable llamado {\tt z\_ejemplo}.

\sectiontext{white}{black}{CARGANDO FICHEROS}

Ya sabemos como generar ficheros comprimidos. Ahora solo tenemos que
saber como leerlos de nuevo desde nuestros programas.

Si os cuento que existe una funci�n llamada {\em gzgets}, seguro que
la mayor�a no necesitar�a saber m�s. Pero por si hay alg�n despistado
en la sala, ah� va un ejemplillo de uso.

{\small
\begin{verbatim}
#include <stdio.h>
#include <zlib.h>

int main ()
{
	gzFile f;
	char   line[256];

	f = gzopen (``mi_fichero.txt.gz'', ``rt'');
	while (!gzeof (f))
	{
	   gzgets (f, line, 256);
	   printf (``%s\n'', line);
	}
	gzclose(f);
}
\end{verbatim}
}

Bastante sencillo no?. As� que ya podemos hacer que nuestros programas
graben sus ficheros de texto en formato comprimido y recuperarlos
posteriormente sin m�s.

\sectiontext{white}{black}{Y QUE M�S?}

Pues para los m�s curiosos que quieran sacarle todo el jugo a esta
librer�a, lo mejor que pueden hacer es mirarse el fichero {\em zlib.h}
que normalmente estar� en el directorio {\tt /usr/include}.

Este fichero contiene todos los prototipos y estructuras de datos
utilizados por la librer�a con amplios comentarios para cada una de
ellas. 

Lo mejor es empezar por el final, donde encontrareis las funciones que
hemos visto en los ejemplos anteriores, y unas cuantas m�s que os
resultar�n muy familiares.

\begin{entradilla}
{\em ``{\color{titlecolor}{La librer�a zlib}} nos facilita el uso de ficheros comprimidos''}
\end{entradilla}

La primera parte del fichero contiene el API de m�s bajo nivel con el
que controlar los par�metros de compresi�n y comprimir/descomprimir
datos en buffers de memoria, lo cual puede ser �til en algunas
circunstancias.

Finalmente, recordad que para poder compilar estos ejemplos necesit�is el paquete de desarrollo {\em zlib} que incluye el fichero de cabecera {\tt zlib.h} que hemos utilizado.

Hasta el pr�ximo n�mero.

\raggedcolumns
\pagebreak

\end{multicols}

\pagebreak
