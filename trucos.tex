% Este fichero es parte del N�mero 1 de la Revista Occam's Razor
% Revista Occam's Razor N�mero 1
%
% (c) 2007, Occam's Razor.
% Contenido disponible bajo licencia Reconocimiento-No comercial-Compartir bajo la misma licencia 2.5 Espa�a de Creative Commons. Para ver una copia de esta licencia, visite http://creativecommons.org/licenses/by-nc-sa/2.5/es/ o envie una carta a Creative Commons, 559 Nathan Abbott Way, Stanford, California 94305, USA.
% 

\rput(13,-22){\resizebox{!}{8cm}{{\epsfbox{trick.eps}}}}
\rput(1,-2){\resizebox{!}{5cm}{{\epsfbox{tophat.eps}}}}
\begin{flushright}
\msection{red}{black}{0.1}{TRUCOS}

\mtitle{6cm}{Con un par... de l�neas}

\msubtitle{8cm}{Chuletillas para hacer cosas m� r�pido}

{\sf por Tamariz el de la Perd�z}

{\psset{linecolor=black,linestyle=dotted}\psline(-10,0)}

\end{flushright}

\vspace{4mm}

\begin{multicols}{2}
\raggedcolumns


\sectiontext{white}{black}{PROCESANDO TEXTO CON PERL EN UNA L�NEA}
\hrule
\vspace{2mm}

Aunque el comando grep funciona perfectamente, puede ser �til
simularlo utilizando una l�nea de c�digo Perl.


\vspace{2mm}

\hrule
{\scriptsize
{\begin{lstlisting}{}
perl -e 'while (<>) {print if /hola/;}' mi_fichero
\end{lstlisting}
}
}
\hrule
\vspace{2mm}

O de forma m�s breve utilizando el flag -n que simplemente comparando
estos dos ejemplos sabr�is qu� hace. 

\vspace{2mm}
\hrule
{\scriptsize
{\begin{lstlisting}{}
perl -ne 'print if /hola/;' mi_fichero
\end{lstlisting}
}
}
\hrule
\vspace{2mm}



Vamos con un ejemplo un poco m�s �til. Supongamos que tenemos un fichero con datos ordenados en columnas y queremos quedarnos solamente con la primera (el valor de ordenadas) y la tercera, digamos que para hacer una representaci�n gr�fica solamente de esos datos. El siguiente script:

\vspace{2mm}

\hrule
{\scriptsize
{\begin{lstlisting}{}
perl -e 'while (<>) {@v=split; 
> print "$v[0]\t$v[2]\n"}' mi_fichero
\end{lstlisting}
}
}
\hrule
\vspace{2mm}

Aunque lo podr�amos haber hecho con \texttt{awk} con una l�nea como

{\scriptsize\verb!cat mi_fichero | awk -e '{print \$1,\$2}'!}


\sectiontext{white}{black}{CREAR IMAGEN CD Y ACCEDER A EL}
\hrule
\vspace{2mm}

El siguiente truco nos permite generar una imagen exacta de un CD y acceder a ella. Las siguientes l�neas hacen el trabajo poniendo el contenido el CD en el directorio /mnt/temp.

\vspace{2mm}


\hrule
{\scriptsize
{\begin{lstlisting}{}
# dd if=/dev/cdrom of=mi_imagen.iso
# mount -o loop mi_imagen.iso /mnt/tmp
# ...
# umount /mnt/tmp
\end{lstlisting}
}
}
\hrule
\vspace{2mm}

Recuerda que debes ser root para ejecutar los comandos del ejemplo 3 y no olvides desmontar el dispositivo cuando hayas terminado con �l.


\sectiontext{white}{black}{MANEJAR CARACTERES DE CONTROL EN VIM}
\hrule
\vspace{2mm}

En ocasiones es necesario manejar caracteres de control dentro de ficheros de texto, por ejemplo, para insertar o sustituir tabuladores. La forma de introducir caracteres como el tabulador en el modo comando del vim es pulsar la combinaci�n de teclas \texttt{CONTROL} + V y luego pulsar la tecla del car�cter que se desea utilizar (return, bs, TAB,...). 

\vspace{6mm}

\sectiontext{white}{black}{GENERAR GR�FICOS A PARTIR DE FICHEROS DE TEXTO}
\hrule
\vspace{2mm}

A partir de un fichero de texto que contenga una columna de datos, podemos obtener r�pidamente una representaci�n gr�fica de los mismos utilizando la herramienta \texttt{gnuplot} utilizando los siguientes comandos:

\vspace{2mm}

\hrule
{\scriptsize
{\begin{lstlisting}{}
# wc -l text.dat
25
# gnuplot
gnuplot> plot [t=1:25] "test.txt" using ($2)
\end{lstlisting}
}
}
\hrule
\vspace{2mm}
%$


Si nuestro fichero tuviera dos columnas en las que la primera
representa los valores de abscisas, la siguiente secuencia de
instrucciones gnuplot mostrar�a la gr�fica. Adem�s, en este caso, los
distintos puntos se unir�n utilizando l�neas rectas (par�metro
\texttt{with lines}). 

\vspace{2mm}

\hrule
{\scriptsize
{\begin{lstlisting}{}
# wc -l text.dat
25
# gnuplot
gnuplot> plot "test.txt" using ($1):($2) with lines
\end{lstlisting}
}
}
\hrule
\vspace{2mm}


%% Call for tricks

\vspace{2mm}

\colorbox{introcolor}{
\begin{minipage}{.9\linewidth}{
\textbf{\textsf{Env�a tus trucos}}

\vspace{1mm}

\textsf{Puedes enviarnos esos trucos que usas a diario para compartirlos con el resto de lectores a la direcci�n: }

\vspace{2mm}

\texttt{occams-razor@uvigo.es}
}
\end{minipage}
}

\raggedcolumns
\pagebreak

\vspace{6cm}
\end{multicols}

\pagebreak
